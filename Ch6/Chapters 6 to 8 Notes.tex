\documentclass[11pt]{article}

\usepackage{sectsty}
\usepackage{siunitx}
\usepackage{graphicx}
\usepackage{amsmath}
\usepackage{amsthm}
\usepackage{float}
\usepackage{outlines}
\usepackage{amssymb}
\usepackage{caption}
\usepackage{subcaption}
\let\oldsi\si
\renewcommand{\si}[1]{\oldsi[per-mode=reciprocal-positive-first]{#1}}
\usepackage{enumitem}
\newcommand{\degsym}{^{\circ}}
\newcommand{\Mod}[1]{\ (\mathrm{mod}\ #1)}
\usepackage{hyperref}
\hypersetup{
    colorlinks,
    citecolor=black,
    filecolor=black,
    linkcolor=black,
    urlcolor=black
}
\newcommand{\lb}{\\[8pt]}
\newenvironment*{cell}[1][]{\begin{tabular}[c]{@{}c@{}}}{\end{tabular}}
\newcommand{\img}[3]{\begin{center}
  \begin{figure}[H]
    \centering
    \includegraphics[width=#2\textwidth]{#1}
    \caption{#3}
    \label{fig:fig1}
  \end{figure}
\end{center}}
\newcommand{\doubleimg}[4]{\begin{center}
  \begin{figure}[H]
    \centering
    \begin{subfigure}{.45\textwidth}
      \centering
      \includegraphics[width=1\linewidth]{#1}
      \caption{#2}
      \label{fig:sub1}
    \end{subfigure}
    \begin{subfigure}{.45\textwidth}
      \centering
      \includegraphics[width=1\linewidth]{#3}
      \caption{#4}
      \label{fig:sub2}
    \end{subfigure}
  \end{figure}
\end{center}}



\title{Math AA HL at KCA - Chapters 6 to 8 Notes}
\author{Tim Bao 2023-2025}
\date{February 25, 2024}

\begin{document}

\maketitle
\pagebreak
\tableofcontents
\pagebreak

\section{Quadratics}

A quadratic function can be in any of the following three forms

\begin{center}
  \begin{tabular}{c|c|c}
    Form     & General expression         & Features                                  \\ \hline
    Standard & $ax^2 + bx + c$            & $c$ = $y$-intercept                       \\
    Vertex   & $a(x - h)^2 + k$           & $(h, k)$ vertex; $x = h$ line of symmetry \\
    Factored & $a(x - \alpha)(x - \beta)$ & $\alpha, \beta$ roots                     \\
  \end{tabular}
\end{center}

\noindent The leading coefficient $a$ determines the concavity of the function
\begin{itemize}
  \item $a > 0 \implies$ the function is convex (tends to $\infty$)
  \item $a < 0 \implies$ the function is concave (tends to $-\infty$)
\end{itemize}

\pagebreak

\section{Sketching Cubics and Higher Degree Graphs}

A polynomial of degree $n$ can have at most $n$ distinct roots. \lb
The following features are to be considered when sketching a polynomial $f(x)$ with leading coefficient $a$

\doubleimg{figs/cubic1.png}{$a > 0$}{figs/cubic2.png}{$a < 0$}

\begin{outline}[enumerate]
  \1 Behaviors as $x\to \pm\infty$; for a cubic:
  \2 $a > 0 \implies$ as $x\to\pm\infty$, $y\to\pm\infty$
  \2 $a < 0 \implies$ as $x\to\pm\infty$, $y\to\mp\infty$
  \1 If the roots are $\alpha, \beta, \gamma$, then the $x$-intercepts are $(\alpha, 0)$, $(\beta, 0)$, $(\gamma, 0)$.
  \1 The $y$-intercept, this can be found by $f(0)$
  \1 The turning point(s), if applicable and necessary.
\end{outline}

\pagebreak

\section{Mappings}

A \textbf{relation} describes how values are mapped from one set onto another.\lb
In the case of functions, values from the \textbf{domain} are mapped onto values in the \textbf{range}.\lb
There are four types of mapping
\img{figs/mapping.jpeg}{0.4}{Mappings}

\noindent For a relation to be a function, it has to be either one-to-one or many-to-one. This is because each $x$ value inputted into the function must be associated with a single $y$ value.


\subsection{Interval Notations}

An opening/closing square bracket represents an inclusive end.\lb
An opening/closing parenthesis represents an exclusive end.\lb
E.g.
\begin{itemize}
  \item $x\in [a, b) \iff a \le x < b$
  \item $x\in (-\infty, h] \iff x \le h$
\end{itemize}


\pagebreak


\subsection{Domain and Range}

The \textbf{range} is the set of $x$ values that can go into a function, and the \textbf{range} is the set of $y$ values that can come out of a function.\lb
Notations for expressing a domain
\begin{enumerate}
  \item $x\in [a, b]$
  \item $a \le x \le b$
  \item $R = \{x \mid a \le x \le b \}$
\end{enumerate}
Notations for expressing a range
\begin{enumerate}
  \item $y\in [a, b]$
  \item $f(x)\in [a, b]$
  \item $a \le f(x) \le b$
  \item $D = \{y \mid a \le y \le b \}$
\end{enumerate}

\pagebreak

\subsection{Restricting the Domain}

\doubleimg{figs/isfn.png}{Function}{figs/nofn.png}{Not a function}

\noindent A relation is not a function if it does not pass the \textbf{vertical line test}, which states that: If \textit{at any point} in the domain of a relation, a vertical line intersects the graph at two or more points, then it is not a function.\lb
For such a relation to be a function, the domain needs to be restricted such that the new domain passes the vertical line test.

\pagebreak

\section{Composite Functions}

$$(f \circ g)(x) = f(g(x))$$

Properties
\begin{itemize}
  \item Associativity $f\circ (g\circ h) = (f\circ g)\circ h$
  \item $(f\circ g)^{-1} = g^{-1} \circ f^{-1}$
\end{itemize}

\section{Inverse Functions}
The inverse function of a function $f(x)$ is denoted $f^{-1}(x)$, it is a reflection in the line $y = x$.\lb
A \textbf{self-inverse} function is its own inverse, i.e. $f(x) \equiv f^{-1}(x)$. E.g. $y = \dfrac{1}{x}$ is a self-inverse function.\lb
Implications:
\begin{itemize}
  \item $(f^{-1} \circ f)(x) = x$ and $(f \circ f^{-1})(x) = x$
  \item The \textit{domain} of $f$ is the \textit{range} of $f^{-1}$, and vice versa
  \item The intersections of $f(x)$ and $f^{-1}(x)$ lie on the line $y = x$ and can be found by $f(x) = x$ or $f^{-1}(x) = x$
\end{itemize}
To find an expression for the inverse function, exchange the $x$'s and $y$'s, then make the new $y$ the subject.\lb
A function has an inverse if and only if it is \textbf{injective} over its domain, i.e. it is a one-to-one mapping over its domain. This can be examined using the \textit{horizontal line test}. \lb
Sometimes to restrict the domain such that there is an inverse, the turning point is used.

\pagebreak

\section{Even and Odd Functions}

\begin{outline}[enumerate]
  \1 Even functions: $f(-x) = f(x)$; symmetrical about the $y$-axis
  \2 $f(x)$ has only even powers of $x$
  \1 Odd functions: $f(-x) = -f(x)$; rotational symmetry of $180\degsym$ about the origin
  \2 $f(x)$ has only odd powers of $x$
\end{outline}

\section{Function Transformations and Descriptions}

\begin{center}
  \bgroup
  \def\arraystretch{1.5}
  \begin{tabular}{|c|c|}
    \hline
    $-f(x)$    & Reflection in the $x$-axis                      \\ \hline
    $f(-x)$    & Reflection in the $y$-axis                      \\ \hline
    $f(x + a)$ & Translation to the left by $a$ units            \\ \hline
    $f(x - a)$ & Translation to the right by $a$ units           \\ \hline
    $f(kx)$    & Horizontal stretch by a factor of $\frac{1}{k}$ \\ \hline
    $f(x) + b$ & Upward translation by $b$ units                 \\ \hline
    $f(x) - b$ & Downward translation by $b$ units               \\ \hline
    $kf(x)$    & Vertical stretch by a factor of $k$             \\ \hline
  \end{tabular}
  \egroup
\end{center}

\pagebreak

\section{Special Functions and Their Graphs}
Sketching these functions involves looking at
\begin{enumerate}
  \item $x$ and $y$ intercepts
  \item horizontal and vertical asymptotes
  \item where $f(x)$ tends to as $x\to\pm\infty$
  \item sometimes the turning points too
\end{enumerate}

\subsection{Rational/Reciprocal Functions}

They take the general form of $$f(x) = \frac{ax + b}{cx + d}$$ and have the following properties
\begin{itemize}
  \item $x$-intercept = $(-\dfrac{b}{a}, 0)$; when $y = 0$ and $a\not = 0$
  \item $y$-intercept = $(0, \dfrac{b}{d})$; when $x = 0$ and $d\not = 0$
  \item vertical asymptote $x = -\dfrac{d}{c}$; when the denominator is 0
  \item horizontal asymptote $y = \dfrac{a}{c}$; cancel out the b, d, and $x$'s, as the limit is obtained for very large values of $x$.
\end{itemize}

\pagebreak

\subsection{Exponential Functions}

They can take the form of $$f(x)=ka^{bx} + c,\,\,\, a > 0$$ and have the properties
\begin{enumerate}
  \item $y$-intercept $(k + c, 0)$
  \item Horizontal asymptote $y = c$
  \item No vertical asymptote
  \item $b > 0 \implies \lim\limits_{x\to\infty}f(x) = \infty$; growth
  \item $b < 0 \implies \lim\limits_{x\to-\infty}f(x) = \infty$; decay
\end{enumerate}


\subsection{Logarithmic Functions}

They are the reflections of exponential functions in $y = x$. They have vertical asymptotes but not horizontal asymptotes.

\pagebreak

\subsection{Modulus Functions}

Arithmetic rules

\begin{enumerate}
  \item $|-x| = |x|$
  \item $|x|^2 = x^2$
  \item $|xy| = |x||y|$
  \item $\left|\dfrac{x}{y}\right| = \dfrac{|x|}{|y|}$
\end{enumerate}

\noindent These functions are formally defined as

$$
  |f(x)| = \left\{
  \begin{array}{ll}
    f(x)  & \mbox{if } f(x) \geq 0 \\
    -f(x) & \mbox{if } f(x) < 0
  \end{array}
  \right.
$$

\noindent The graph is obtained by reflecting all parts below the $y$-axis through the $y$-axis so that they are all above it now.\lb
In general, there are three ways of solving modulus equations/inequalities; solving manually will require the results to be substituted back into the equation to validate the answers and discard any invalid ones
\begin{enumerate}
  \item Split into cases. e.g. for $|a| = 3$, solve when $a = 3$ and also $a = -3$, then validate the results
  \item If both sides are in absolute value, or only one side is and the other is a constant (e.g. $|a| = |b|$ or $|a| = 3$), then square both sides and solve from there
  \item Sketch graphs and find intersections
  \item Use the G.D.C
\end{enumerate}
The transformation of absolute value functions can be \textit{done step by step} and it is normally easier to visualize this way.

\subsection{Graphs of \texorpdfstring{$[f(x)]^2$}{f(x) squared}}

\begin{center}
  \def\arraystretch{2}%
  \begin{tabular}{c|c}
    Feature of $f(x)$               & Feature of $[f(x)]^2$                    \\ \hline
    Parts where $y < 0$             & $y > 0$                                  \\
    this i the p
    $(a, 0)$ is an x-intercept      & $(a, 0)$ local minimum                   \\
    $(0, b)$ is a y-intercept       & $(0, b^2)$ y-intercept                   \\
    $x=a$ is a vertical asymptote   & unchanged                                \\
    $y=b$ is a horizontal asymptote & $y = b^2$ becomes a horizontal asymptote \\
    $y\to \pm\infty$                & $y \to \infty$                           \\
  \end{tabular}
\end{center}


\subsection{Graphs of \texorpdfstring{$\frac{1}{f(x)}$}{the reciprocal of f(x)}}

\begin{center}
  \def\arraystretch{2}%
  \begin{tabular}{c|c}
    Feature of $f(x)$                      & Feature of $\frac{1}{f(x)}$                 \\ \hline
    $x=a$ is an x-intercept                & $x = a$ is a vertical asymptote             \\
    $(0, b),\, b\not=0$ is a y-intercept   & $(0, \frac{1}{b})$ becomes the y-intercept  \\
    $x=a$ is a vertical asymptote          & x-intercept at $(a, 0)$                     \\
    $y=b$ is a horizontal asymptote        & $y = \frac{1}{b}$ is a horizontal asymptote \\
    $y=0$ is a horizontal asymptote        & $y\to \pm \infty$                           \\
    $y\to \pm\infty$                       & $y = 0$ is a horizontal asymptote           \\
    $(a, b),\, b\not=0$ is a turning point & $(a, \frac{1}{b})$ is a turning point       \\
  \end{tabular}
\end{center}

\end{document}