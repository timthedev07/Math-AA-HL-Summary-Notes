\documentclass[11pt]{article}

\usepackage{sectsty}
\usepackage{siunitx}
\usepackage{graphicx}
\usepackage{amsmath}
\usepackage{amsthm}
\usepackage{float}
\usepackage{outlines}
\usepackage{amssymb}
\usepackage{caption}
\usepackage{subcaption}
\let\oldsi\si
\renewcommand{\si}[1]{\oldsi[per-mode=reciprocal-positive-first]{#1}}
\usepackage{enumitem}
\newcommand{\degsym}{^{\circ}}
\newcommand{\Mod}[1]{\ (\mathrm{mod}\ #1)}
\newcommand{\lb}{\\[8pt]}
\newenvironment*{cell}[1][]{\begin{tabular}[c]{@{}c@{}}}{\end{tabular}}
\newcommand{\img}[3]{\begin{center}
  \begin{figure}[H]
    \centering
    \includegraphics[width=#2\textwidth]{#1}
    \caption{#3}
    \label{fig:fig1}
  \end{figure}
\end{center}}
\newcommand{\doubleimg}[4]{\begin{center}
  \begin{figure}[H]
    \centering
    \begin{subfigure}{.45\textwidth}
      \centering
      \includegraphics[width=1\linewidth]{#1}
      \caption{#2}
      \label{fig:sub1}
    \end{subfigure}
    \begin{subfigure}{.45\textwidth}
      \centering
      \includegraphics[width=1\linewidth]{#3}
      \caption{#4}
      \label{fig:sub2}
    \end{subfigure}
  \end{figure}
\end{center}}

\title{Math AA HL at KCA - Chapters 6 to 9 Notes}
\author{Tim Bao 2023-2025}
\date{February 25, 2024}

\begin{document}

\maketitle
\pagebreak

\section{Quadratics}

A quadratic function can be in any of the following three forms

\begin{center}
  \begin{tabular}{c|c|c}
    Form     & General expression         & Features                                  \\ \hline
    Standard & $ax^2 + bx + c$            & $c$ = $y$-intercept                       \\
    Vertex   & $a(x - h)^2 + k$           & $(h, k)$ vertex; $x = h$ line of symmetry \\
    Factored & $a(x - \alpha)(x - \beta)$ & $\alpha, \beta$ roots                     \\
  \end{tabular}
\end{center}

\noindent The leading coefficient $a$ determines the concavity of the function
\begin{itemize}
  \item $a > 0 \implies$ the function is convex (tends to $\infty$)
  \item $a < 0 \implies$ the function is concave (tends to $-\infty$)
\end{itemize}

\pagebreak

\section{Sketching Cubics and Higher Degree Graphs}

A polynomial of degree $n$ can have at most $n$ distinct roots. \lb
The following features are to be considered when sketching a polynomial $f(x)$ with leading coefficient $a$

\doubleimg{figs/cubic1.png}{$a > 0$}{figs/cubic2.png}{$a < 0$}

\begin{outline}[enumerate]
  \1 Behaviors as $x\to \pm\infty$; for a cubic:
  \2 $a > 0 \implies$ as $x\to\pm\infty$, $y\to\pm\infty$
  \2 $a < 0 \implies$ as $x\to\pm\infty$, $y\to\mp\infty$
  \1 If the roots are $\alpha, \beta, \gamma$, then the $x$-intercepts are $(\alpha, 0)$, $(\beta, 0)$, $(\gamma, 0)$.
  \1 The $y$-intercept, this can be found by $f(0)$
  \1 The turning point(s), if applicable and necessary.
\end{outline}

\pagebreak

\section{Mappings}

A \textbf{relation} describes how values are mapped from one set onto another.\lb
In the case of functions, values from the \textbf{domain} are mapped onto values in the \textbf{range}.\lb
There are four types of mapping
\img{figs/mapping.jpeg}{0.4}{Mappings}

\noindent For a relation to be a function, it has to be either one-to-one or many-to-one. This is because each $x$ value inputted into the function must be associated with a single $y$ value.


\subsection{Interval Notations}

An opening/closing square bracket represents an inclusive end.\lb
An opening/closing parenthesis represents an exclusive end.\lb
E.g.
\begin{itemize}
  \item $x\in [a, b) \iff a \le x < b$
  \item $x\in (-\infty, h] \iff x \le h$
\end{itemize}


\pagebreak


\subsection{Domain and Range}

The \textbf{range} is the set of $x$ values that can go into a function, and the \textbf{range} is the set of $y$ values that can come out of a function.\lb
Notations for expressing a domain
\begin{enumerate}
  \item $x\in [a, b]$
  \item $a \le x \le b$
  \item $R = \{x \mid a \le x \le b \}$
\end{enumerate}
Notations for expressing a range
\begin{enumerate}
  \item $y\in [a, b]$
  \item $f(x)\in [a, b]$
  \item $a \le f(x) \le b$
  \item $D = \{y \mid a \le y \le b \}$
\end{enumerate}

\pagebreak

\subsection{Restricting the Domain}

\doubleimg{figs/isfn.png}{Function}{figs/nofn.png}{Not a function}

\noindent A relation is not a function if it does not pass the \textbf{vertical line test}, which states that: If \textit{at any point} in the domain of a relation, a vertical line intersects the graph at two or more points, then it is not a function.\lb
For such a relation to be a function, the domain needs to be restricted such that the new domain passes the vertical line test.

\pagebreak

\section{Composite Functions}

$$(f \circ g)(x) = f(g(x))$$

Properties
\begin{itemize}
  \item Associativity $f\circ (g\circ h) = (f\circ g)\circ h$
  \item $(f\circ g)^{-1} = g^{-1} \circ f^{-1}$
\end{itemize}

\section{Inverse Functions}
The inverse function of a function $f(x)$ is denoted $f^{-1}(x)$, it is a reflection in the line $y = x$.\lb
A \textbf{self-inverse} function is its own inverse, i.e. $f(x) \equiv f^{-1}(x)$. E.g. $y = \dfrac{1}{x}$ is a self-inverse function.\lb
Implications:
\begin{itemize}
  \item $(f^{-1} \circ f)(x) = x$ and $(f \circ f^{-1})(x) = x$
  \item The \textit{domain} of $f$ is the \textit{range} of $f^{-1}$, and vice versa
  \item The intersections of $f(x)$ and $f^{-1}(x)$ lie on the line $y = x$ and can be found by $f(x) = x$ or $f^{-1}(x) = x$
\end{itemize}
To find an expression for the inverse function, exchange the $x$'s and $y$'s, then make the new $y$ the subject.

\section{Even and Odd Functions}

\begin{outline}[enumerate]
  \1 Even functions: $f(-x) = f(x)$; symmetrical about the $y$-axis
  \2 $f(x)$ has only even powers of $x$
  \1 Odd functions: $f(-x) = -f(x)$; rotational symmetry of $180\degsym$ about the origin
  \2 $f(x)$ has only odd powers of $x$
\end{outline}


\end{document}