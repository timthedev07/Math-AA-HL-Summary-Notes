\documentclass[11pt]{article}

\usepackage{sectsty}
\usepackage{siunitx}
\usepackage{graphicx}
\usepackage{amsmath}
\usepackage{amsthm}
\usepackage{float}
\usepackage{outlines}
\usepackage{caption}
\usepackage{subcaption}
\usepackage{amssymb}
\let\oldsi\si
\renewcommand{\si}[1]{\oldsi[per-mode=reciprocal-positive-first]{#1}}
\usepackage{enumitem}
\newcommand{\degsym}{^{\circ}}
\newcommand{\Mod}[1]{\ (\mathrm{mod}\ #1)}
\newcommand{\lb}{\\[8pt]}
\newenvironment*{cell}[1][]{\begin{tabular}[c]{@{}c@{}}}{\end{tabular}}
\newcommand{\img}[3]{\begin{center}
  \begin{figure}[H]
    \centering
    \includegraphics[width=#2\textwidth]{#1}
    \caption{#3}
    \label{fig:fig1}
  \end{figure}
\end{center}}
\newcommand{\doubleimg}[4]{\begin{center}
  \begin{figure}[H]
    \centering
    \begin{subfigure}{.45\textwidth}
      \centering
      \includegraphics[width=1\linewidth]{#1}
      \caption{#2}
      \label{fig:sub1}
    \end{subfigure}
    \begin{subfigure}{.45\textwidth}
      \centering
      \includegraphics[width=1\linewidth]{#3}
      \caption{#4}
      \label{fig:sub2}
    \end{subfigure}
  \end{figure}
\end{center}}

\title{Math AA HL at KCA - Chapter 5 Notes}
\author{Tim Bao 2023-2025}
\date{February 25, 2024}

\begin{document}

\maketitle
\pagebreak

\section{Forms of Complex Numbers and Operations}

\subsection{Cartesian Form}

$$z = a + bi$$
where
\begin{itemize}
  \item $a = \text{Re}{(z)}$
  \item $b = \text{Im}{(z)}$
\end{itemize}

\noindent The complex conjugate of $z$ will then be

$$z^* = a - bi$$

\noindent The addition and subtracting of complex numbers in the Cartesian form are done by adding/subtracting the real components and the imaginary components respectively.\lb
Division insolves multiplying by the \textbf{complex conjugate} of the denominator throughout
$$\frac{a + bi}{c + di} = \frac{(a + bi)(c - di)}{(c + di)(c - di)}$$ this is so that the imaginary component will be canceled out (similar to rationalizing denominators)

\pagebreak

\subsection{Modulus-Argument Form}

$$z = r\left(\cos\theta + i\sin\theta\right)$$
where
\begin{itemize}
  \item $r = |z|$ = modulus; the straight line distance from the origin to $z$ on the Argand diagram
  \item $\theta = \arg(z)$ = argument; the angle measured counter-clockwise from the positive $x$-axis.
\end{itemize}

\subsubsection*{Operations on moduli and arguments}

Suppose $z$ and $w$ are two complex numbers, then
\begin{align*}
  |zw|          & = |z||w|          \\
  |\frac{z}{w}| & = \frac{|z|}{|w|} \\
\end{align*}
\begin{align*}
  \arg(zw)          & = \arg(z) + \arg(w) \\
  \arg(\frac{z}{w}) & = \arg(z) - \arg(w) \\
\end{align*}

\subsubsection*{Operations on modulus-argument}
Multiplication:
$$\left(r_1(\cos \alpha + i\sin\alpha)\right)\left(r_2(\cos \beta + i\sin\beta)\right) = r_1r_2\left(\cos(\alpha + \beta) + i\sin(\alpha + \beta)\right)$$
Division:
$$\frac{r_1\left(\cos\alpha + i\sin\alpha\right)}{r_2\left(\cos \beta + i\sin\beta\right)} = \frac{r_1}{r_2}\left(\cos(\alpha - \beta) + i\sin(\alpha - \beta)\right)$$

\pagebreak

\subsection{Exponential/Euler's Form}

$$z = re^{i\theta}$$
where
\begin{itemize}
  \item $r = |z|$
  \item $\theta = \arg(z)$
\end{itemize}

\noindent Multipliation and division:

$$zw = r_1r_2e^{i(\alpha + \beta)}$$

$$\frac{z}{w} = \frac{r_1}{r_2}e^{i(\alpha - \beta)}$$

\pagebreak

\section{The Argand Diagram}

\doubleimg{figs/1.png}{The Argand diagram}{figs/roots.jpeg}{5th roots}

\noindent Key points
\begin{outline}[enumerate]
  \1 For a complex number $z = a + bi$
  \2 $|z| = \sqrt{a^2 + b^2}$
  \2 $\arg(z) = \arctan(\frac{b}{a})$, but a further step of using other information such as the quadrant is required to determine the argument, since $\tan(\theta) = \tan(\pi + \theta)$.
  \1 The real axis is a line of symmetry for the complex solutions of a polynomial equation; this is because complex roots are \textbf{pairwise conjugate}.
  \1 The $n$-th roots of a complex number \textbf{have the same modulus} and \textbf{equally divide a full turn}. This leads to implications such as
  \2 The roots join to form a circle with a radius equal to their mutual modulus.
  \2 By joining the edges, a regular $n$-sided polygon is formed.
\end{outline}

\end{document}