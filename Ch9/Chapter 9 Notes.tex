\documentclass[11pt]{article}

\usepackage{sectsty}
\usepackage{siunitx}
\usepackage{graphicx}
\usepackage{amsmath}
\usepackage{amsthm}
\usepackage{float}
\usepackage{outlines}
\usepackage{amssymb}
\usepackage{outlines}
\usepackage{caption}
\usepackage{subcaption}
\let\oldsi\si
\renewcommand{\si}[1]{\oldsi[per-mode=reciprocal-positive-first]{#1}}
\usepackage{enumitem}
\newcommand{\degsym}{^{\circ}}
\newcommand{\Mod}[1]{\ (\mathrm{mod}\ #1)}
\usepackage{hyperref}
\hypersetup{
    colorlinks,
    citecolor=black,
    filecolor=black,
    linkcolor=black,
    urlcolor=black
}
\newcommand{\lb}{\\[8pt]}
\newenvironment*{cell}[1][]{\begin{tabular}[c]{@{}c@{}}}{\end{tabular}}
\newcommand{\img}[3]{\begin{center}
  \begin{figure}[H]
    \centering
    \includegraphics[width=#2\textwidth]{#1}
    \caption{#3}
    \label{fig:fig1}
  \end{figure}
\end{center}}
\newcommand{\doubleimg}[4]{\begin{center}
  \begin{figure}[H]
    \centering
    \begin{subfigure}{.45\textwidth}
      \centering
      \includegraphics[width=1\linewidth]{#1}
      \caption{#2}
      \label{fig:sub1}
    \end{subfigure}
    \begin{subfigure}{.45\textwidth}
      \centering
      \includegraphics[width=1\linewidth]{#3}
      \caption{#4}
      \label{fig:sub2}
    \end{subfigure}
  \end{figure}
\end{center}}



\title{Math AA HL at KCA - Chapter 9 Notes}
\author{Tim Bao 2023-2025}
\date{February 26, 2024}

\begin{document}

\maketitle
\pagebreak
\tableofcontents
\pagebreak

\section{Partial Fractions}

When splitting a rational function into partial fractions, the denominator must have a \textbf{strictly higher degree} than the numerator. If not, the rational function has to be rewritten.

\noindent Example: $$\frac{3x^2 + 2x - 11}{(x-1)(x+2)} \equiv 3 - \frac{x+5}{(x - 1)(x + 2)}$$

\noindent In general, a rational function that cannot be decomposed yet is to be rewritten in the form of $$p(x) + \frac{q(x)}{r(x)}$$
where
\begin{itemize}
  \item $\deg(p) = \deg(\text{numerator}) - \deg(\text{denominator})$
  \item $\deg(q) < \deg(r)$
\end{itemize}

\subsection{Case 1 - Linear Factors}

A fraction with only linear factors $Q_1(x), Q_2(x), \dots, Q_n(x)$ in the denominator is decomposed into a sum of fractions where $A, B, ...$ are constants:
$$\frac{P(x)}{Q_1(x)Q_2(x)\dots Q_n(x)} \equiv \frac{A}{Q_1(x)} + \frac{B}{Q_2(x)} + \dots + \frac{Z}{Q_n(x)}$$
The first way of decomposition:
\begin{enumerate}
  \item Multiply both sides by $Q_1(x)Q_2(x)\dots Q_n(x)$
  \item Expand
  \item Compare coefficients to form a system of equations and solve for the values
\end{enumerate}
The second way of decomposition:
\begin{enumerate}
  \item Multiply both sides by $Q_1(x)Q_2(x)\dots Q_n(x)$
  \item Choose and substitute values of $x$ that can eliminate one of the unknowns, i.e. $x \mid Q_i(x) = 0$ to form an equation.
  \item Repeat this for other $Q(x)$ until a solvable system is formed.
\end{enumerate}

\subsection{Case 2 - Repeated Linear Factors}

When a linear factor $Q(x)$ is raised to the $n$th power, there should be one decomposed fraction for each power of $Q(x)$.\lb
Example:
$$\frac{-x^2 + 6x +11}{(x - 1)^2 (x+3)} \equiv \frac{A}{x - 1} + \frac{B}{(x - 1)^2} + \frac{C}{x + 3} $$


\subsection{Case 3 - Higher Order Factors}

When there is a polynomial $R(x)$ of a higher (than linear) order $n$ in the denominator that cannot be further factored into linear factors, for the corresponding decomposed fraction, the unknown polynomial in its numerator must have a degree of $n-1$.\lb
Example
$$\frac{5x^2 -5x + 12}{(x^2 + 5)(x-1)} \equiv \frac{Ax + B}{x^2 + 5} +\frac{C}{x - 1}$$

\pagebreak

\section{Polynomial Long Division}

\img{longdiv.png}{0.6}{Long division example}

\section{The Factor and Remainder Theorems}

\begin{enumerate}
  \item
        \textit{The remainder theorem:} The remainder of a polynomial $P(x)$ when divided by $(x - \alpha)$ is $P(\alpha)$

  \item \textit{The factor theorem:} A value $\alpha$ is a root of the polynomial $P(x)$ if and only if $(x - \alpha)$ is a factor of $P(x)$. In other words:
        $$P(\alpha) = 0 \iff (x - \alpha) \mid P(x)$$
\end{enumerate}

\pagebreak

\section{Roots of Polynomial}

For any polynomial $$P(x) = ax^n + bx^{n - 1} + ... $$
The sum of the roots is $\dfrac{-b}{a}$.\lb
Let $a_{0}$ be the constant coefficient, then the product of the roots are
\begin{enumerate}
  \item if $P(x)$ has an \textit{even} degree, then $\dfrac{a_{0}}{a}$
  \item if $P(x)$ has an \textit{odd} degree, then $-\dfrac{a_{0}}{a}$
\end{enumerate}



\end{document}