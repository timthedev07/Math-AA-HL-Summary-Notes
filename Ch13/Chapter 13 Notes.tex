\documentclass[11pt]{article}

\usepackage{sectsty}
\usepackage{siunitx}
\usepackage{graphicx}
\usepackage{amsmath}
\usepackage{amsthm}
\usepackage{float}
\usepackage{outlines}
\usepackage{amssymb}
\usepackage{outlines}
\usepackage{caption}
\usepackage{subcaption}
\let\oldsi\si
\renewcommand{\si}[1]{\oldsi[per-mode=reciprocal-positive-first]{#1}}
\usepackage{enumitem}
\newcommand{\degsym}{^{\circ}}
\newcommand{\Mod}[1]{\ (\mathrm{mod}\ #1)}
\usepackage{hyperref}
\hypersetup{
    colorlinks,
    citecolor=black,
    filecolor=black,
    linkcolor=black,
    urlcolor=black
}
\newcommand{\lb}{\\[8pt]}
\newenvironment*{cell}[1][]{\begin{tabular}[c]{@{}c@{}}}{\end{tabular}}
\newcommand{\img}[3]{\begin{center}
  \begin{figure}[H]
    \centering
    \includegraphics[width=#2\textwidth]{#1}
    \caption{#3}
    \label{fig:fig1}
  \end{figure}
\end{center}}
\newcommand{\doubleimg}[4]{\begin{center}
  \begin{figure}[H]
    \centering
    \begin{subfigure}{.45\textwidth}
      \centering
      \includegraphics[width=1\linewidth]{#1}
      \caption{#2}
      \label{fig:sub1}
    \end{subfigure}
    \begin{subfigure}{.45\textwidth}
      \centering
      \includegraphics[width=1\linewidth]{#3}
      \caption{#4}
      \label{fig:sub2}
    \end{subfigure}
  \end{figure}
\end{center}}



\title{Math AA HL at KCA - Chapter 13 Notes}
\author{Tim Bao 2023-2025}
\date{February 26, 2024}

\begin{document}

\maketitle
\pagebreak
\tableofcontents
\pagebreak

\section{Continuity}

A function is continuous if it can be drawn entirely over its domain without "lifting the pen".\lb
The points of discontinuity are a set of points in the domain of a function at which the function is discontinuous.

\pagebreak

\section{The Laws of Limit}

If $\lim\limits_{x\to a}{f(x)} = l$ and $\lim\limits_{x\to a}{g(x)} = m$, then
\begin{itemize}
  \item Additive: $\lim\limits_{x\to a}{f(x) \pm g(x)} = l\pm m$
  \item Multiplicative: $\lim\limits_{x\to a}{f(x) \cdot g(x)} = lm$
  \item Reciprocal: $\lim\limits_{x\to a}\left(\dfrac{f(x)}{g(x)}\right) = \dfrac{l}{m}$
\end{itemize}


\end{document}