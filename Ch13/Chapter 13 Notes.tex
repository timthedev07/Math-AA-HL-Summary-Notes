\documentclass[11pt]{article}

\usepackage{sectsty}
\usepackage{siunitx}
\usepackage{graphicx}
\usepackage{amsmath}
\usepackage{amsthm}
\usepackage{float}
\usepackage{outlines}
\usepackage{amssymb}
\usepackage{outlines}
\usepackage{caption}
\usepackage{subcaption}
\let\oldsi\si
\renewcommand{\si}[1]{\oldsi[per-mode=reciprocal-positive-first]{#1}}
\usepackage{enumitem}
\newcommand{\degsym}{^{\circ}}
\newcommand{\Mod}[1]{\ (\mathrm{mod}\ #1)}
\usepackage{hyperref}
\hypersetup{
    colorlinks,
    citecolor=black,
    filecolor=black,
    linkcolor=black,
    urlcolor=black
}
\newcommand{\lb}{\\[8pt]}
\newenvironment*{cell}[1][]{\begin{tabular}[c]{@{}c@{}}}{\end{tabular}}
\newcommand{\img}[3]{\begin{center}
  \begin{figure}[H]
    \centering
    \includegraphics[width=#2\textwidth]{#1}
    \caption{#3}
    \label{fig:fig1}
  \end{figure}
\end{center}}
\newcommand{\doubleimg}[4]{\begin{center}
  \begin{figure}[H]
    \centering
    \begin{subfigure}{.45\textwidth}
      \centering
      \includegraphics[width=1\linewidth]{#1}
      \caption{#2}
      \label{fig:sub1}
    \end{subfigure}
    \begin{subfigure}{.45\textwidth}
      \centering
      \includegraphics[width=1\linewidth]{#3}
      \caption{#4}
      \label{fig:sub2}
    \end{subfigure}
  \end{figure}
\end{center}}



\title{Math AA HL at KCA - Chapter 13 Notes}
\author{Tim Bao 2023-2025}
\date{February 26, 2024}

\begin{document}

\maketitle
\pagebreak
\tableofcontents
\pagebreak

\section{Continuity}

A function is continuous if it can be drawn entirely over its domain without "lifting the pen".\lb
The points of discontinuity are a set of points in the domain of a function at which the function is discontinuous.\lb
A function $f(x)$ is continuous at a point $\iff$ $\lim\limits_{x\to a}f(x)$ exists $\iff$ $\lim\limits_{x\to a^-}{f(x)} = \lim\limits_{x\to a^+}{f(x)}$

\subsection{Removable and Essential Discontinuities}

There is a \textit{removable discontinuity} at $x = a$ if $f(a)$ does not exist and $\lim\limits_{x\to a}f(x)$ exists such that a newly modified function $g(x)$, satisfies $g(a) = \lim\limits_{x \to a}f(x)$. I.e. a new function can be defined such that the single "hole" would be "filled in" and the function is continuous at that point.\lb
If such a modification cannot be made, then it is an \textit{essential discontinuity}. This is when the two pieces are completely disjointed and not connected by a "hole".

\pagebreak

\section{The Laws of Limit}

If $\lim\limits_{x\to a}{f(x)} = L$ and $\lim\limits_{x\to a}{g(x)} = M$, then
\begin{itemize}
  \item Additive: $\lim\limits_{x\to a}{f(x) \pm g(x)} = L\pm M$
  \item Multiplicative: $\lim\limits_{x\to a}{f(x) \cdot g(x)} = LM$
  \item Reciprocal: $\lim\limits_{x\to a}\left(\dfrac{f(x)}{g(x)}\right) = \dfrac{L}{M}$ if $M \not = 0$
  \item L'Hôpital's: $\lim\limits_{x\to a}\left(\dfrac{f(x)}{g(x)}\right) = \lim\limits_{x\to a}\left(\dfrac{f'(x)}{g'(x)}\right)$
\end{itemize}
\bigskip
Indeterminate forms and the corresponding ways of evaluation
\begin{itemize}
  \item $\dfrac{0}{0} \longrightarrow$ factorizing, lateral limits, or L'Hôpital's rule.
  \item $\dfrac{\infty}{\infty} \longrightarrow$ comparison or L'Hôpital's rule
  \item $\dfrac{1}{0} \longrightarrow$ lateral limits
\end{itemize}

\section{Existence of Limits}
The limit $\lim\limits_{x\to a}{f(x)} = L$ exists $\iff$ $\lim\limits_{x\to a^-}{f(x)} = \lim\limits_{x\to a^+}{f(x)} = L$. I.e. approaching $x = a$ from both positive and negative directions, the function converges to the \textit{same limit} $y = L$.\lb
The function diverges when there is not a limit or the limit is $\infty$.

\pagebreak

\section{Limits at Infinity}

Three cases for $\dfrac{f(x)}{g(x)}$
\begin{enumerate}
  \item $\deg(f) >\deg(g) \implies \lim\limits_{x\to\infty}{\dfrac{f(x)}{g(x)}}= \infty$
  \item $\deg(f) =\deg(g) \implies \lim\limits_{x\to\infty}{\dfrac{f(x)}{g(x)}}$ is the quotient of the coefficients of the highest power of x of both polynomials respectively
  \item $\deg(f) <\deg(g) \implies \lim\limits_{x\to\infty}{\dfrac{f(x)}{g(x)}}= 0$
\end{enumerate}
For the second case, if both $f(x)$ and $g(x)$ are cubics, then, the limit is simply the coefficient of $x^3$ in $f(x)$ divided by the coefficient of $x^3$ in $g(x)$\lb
Limit at infinity of exponentials
$$\lim\limits_{x\to\pm\infty} e^{\mp x} = 0$$

\end{document}