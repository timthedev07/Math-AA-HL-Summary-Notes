\documentclass[11pt]{article}

\usepackage{sectsty}
\usepackage{siunitx}
\usepackage{graphicx}
\usepackage{amsmath}
\usepackage{amsthm}
\usepackage{float}
\usepackage{outlines}
\usepackage{amssymb}
\usepackage{outlines}
\usepackage{caption}
\usepackage{subcaption}
\let\oldsi\si
\renewcommand{\si}[1]{\oldsi[per-mode=reciprocal-positive-first]{#1}}
\usepackage{enumitem}
\newcommand{\degsym}{^{\circ}}
\newcommand{\Mod}[1]{\ (\mathrm{mod}\ #1)}
\usepackage{hyperref}
\hypersetup{
    colorlinks,
    citecolor=black,
    filecolor=black,
    linkcolor=black,
    urlcolor=black
}
\newcommand{\lb}{\\[8pt]}
\newenvironment*{cell}[1][]{\begin{tabular}[c]{@{}c@{}}}{\end{tabular}}
\newcommand{\img}[3]{\begin{center}
  \begin{figure}[H]
    \centering
    \includegraphics[width=#2\textwidth]{#1}
    \caption{#3}
    \label{fig:fig1}
  \end{figure}
\end{center}}
\newcommand{\doubleimg}[4]{\begin{center}
  \begin{figure}[H]
    \centering
    \begin{subfigure}{.45\textwidth}
      \centering
      \includegraphics[width=1\linewidth]{#1}
      \caption{#2}
      \label{fig:sub1}
    \end{subfigure}
    \begin{subfigure}{.45\textwidth}
      \centering
      \includegraphics[width=1\linewidth]{#3}
      \caption{#4}
      \label{fig:sub2}
    \end{subfigure}
  \end{figure}
\end{center}}



\title{Math AA HL at KCA - Chapter 10 to 11 Notes}
\author{Tim Bao 2023-2025}
\date{February 26, 2024}

\begin{document}

\maketitle
\pagebreak
\tableofcontents
\pagebreak

\section{Radians, Arc Length, and Area of a Circle}
Radian to degree conversion
$$\theta \degsym = \frac{180}{\pi}\theta^c$$
Degree to radian conversion
$$\theta^c = \frac{\theta\degsym}{180}\pi$$
Arc length $l$ enclosed by two radii $r$ at an angle $\theta$
$$l = r\theta$$
Area of a sector $$A = \frac{r^2\theta}{2}$$

\section{Sine Rule and Area of a Triangle}

Full sine rule $$2R = \frac{a}{\sin A} = \frac{b}{\sin B} = \frac{c}{\sin C}$$
where
\begin{itemize}
  \item $R$ is the circumradius of the triangle
\end{itemize}

\noindent Given two sides $a, b$ and the angle in between $C$, the area of the triangle is given by $$A = \frac{1}{2}ab\sin C$$

\section{Cosine Rule}

Cosine rule for a side

$$a^2 = b^2 + c^2 - 2bc\cos(A)$$
For an angle
$$\cos A = \frac{b^2 + c^2 - a^2}{2bc}$$

\section{Cones}

\img{cone.png}{0.4}{Cone}

\begin{itemize}
  \item $l^2 = h^2 + r^2$
  \item $A = \pi r^2 + \pi r l$
  \item $V = \frac{1}{3}\pi r^2h$
\end{itemize}

\section{Radian-Degree Special Values}

\img{imagetrigfunctions9.jpg}{0.9}{Conversion Table}

\section{Trigonometric Graphs}

\img{graphs.png}{1}{Graphs}

\noindent Periodicity:
\begin{itemize}
  \item $\tan(\theta + k\pi) = \tan(\theta), k\in \mathbb{Z}$
  \item $\sin(\theta + 2k\pi) = \sin(\theta), k\in \mathbb{Z}$
  \item $\cos(\theta + 2k\pi) = \cos(\theta), k\in \mathbb{Z}$
\end{itemize}

\pagebreak

\subsection{Transformations of the Sine/Cosine Graphs}

$$y = a\sin(b(x - c)) + d,\,\,\,\,\,\, a, b > 0$$
\begin{itemize}
  \item Amplitude = $|a|$
  \item Period = $\dfrac{2\pi}{b}$
  \item Principle axis $y = d$
  \item Maximum $a + d$, minimum $-a + d$
\end{itemize}
It is obtained from the transformations
\begin{itemize}
  \item if $a < 0$ then a reflection in the $x$-axis
  \item vertical stretch by factor $|a|$
  \item horizontal stretch by factor $\dfrac{1}{b}$
  \item a translation through $\begin{pmatrix}c\\d\end{pmatrix}$
\end{itemize}

\section{The Unit Circle}

\img{unitcirc.png}{1}{The unit circle}

\noindent Angles measured counter-clockwise from the positive $x$-axis are positive, and angles measured clockwise from the $x$-axis are negative.

\noindent A "reference angle" is the angle $\theta$ in the first quadrant.
\begin{itemize}
  \item The corresponding angle in the \textit{second quadrant} is $\pi - \theta$
  \item The corresponding angle in the \textit{third quadrant} is $\pi + \theta$
  \item The corresponding angle in the \textit{fourth quadrant} is $- \theta$
\end{itemize}
Each angle's trigonometric ratios have the same \textit{magnitude} as its reference angle; using information about the quadrant will help to determine the signs.

\pagebreak

\section{Trigonometric Identities}
\section{Compound Angle and Half Angle Formulae}

\end{document}